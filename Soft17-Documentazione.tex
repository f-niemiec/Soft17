\documentclass{article}
\usepackage{graphicx} % Required for inserting images
\usepackage{hyperref}
\title{Soft17 - An Agent for Blackjack}
\author{Abbatiello Simone, Nappi Vincenzo , Niemiec Francesco }
\date{January 2025}

\begin{document}

\maketitle

\section{Introduzione}
\subsection{Genesi del progetto}
Durante la seconda parte del corso di Fondamenti di Intelligenza Artificiale, in particolare quella relativa agli argomenti di Machine Learning,  il parallelo tra gli argomenti del corso stesso ed il mondo della statistica è diventato sempre più palese. Abbiamo deciso di sfruttare l'occasione del dover sviluppare un progetto per il corso per approfondire la genuina curisità relativa a questa correlazione.
Detto ciò abbiamo quindi fatto un brainstorming per comprendere quale ambito fissare per unire questi due mondi ancor di più. 
Guardando i documenti messi a disposizione dal docente abbiamo trovato un denominatore comune, ovvero i giochi. 
Dettati dalla curiosità e dalla sua semplicità abbiamo quindi optato per il gioco del Blackjack.
Da ciò nasce l'idea di voler sviluppare un agente intelligente che fosse in grado di poter giocare autonomamente a blackjack.
\section{Descrizione dell'agente}
\subsection{Obiettivi}
Lo scopo del progetto è quello di sviluppare un agente intelligente che sia in grado di eseguire i seguenti compiti:
\begin{itemize}
    \item Selezionare per ogni mano osservabile l'azione più appropriata tra: 
    \begin{itemize}
        \item[Hit]: richiedere la carta successiva;
        \item[Stand]: rimanere con la mano attuale;
        \item[Double]: raddoppiare la puntata iniziale;
        \item[Split]: in caso di due carte uguali poter dividerle in due mani diverse da giocare.
    \end{itemize}
    \item Adattarsi all'ambiente parzialmente osservabile, ovvero prendere decisioni solo sulla base della propria mano e di quella del dealer, senza conoscere il resto del mazzo o le carte rimanenti.
    \item Migliorare le proprie prestazioni nel tempo.
\end{itemize}
\subsection{Specifica dell'ambiente}
Per caratterizzare l'ambiente in cui l'agente andrà ad operare abbiamo utilizzato la specifica PEAS:
P: La misura di prestazione utilizzata per valutare l’operato dell'agente.
E: Descrizione di tutti gli elementi che formano l’ambiente.
A: attuatori che l’agente usa per svolgere le azioni.
S: sensori con i quali riceve i vari input.
In particolare l'ambiente è caratterizzato in questo modo:
\begin{description}
    \item[Performance:] la misura di prestazione dell'agente è la sua capacità di massimizzare il successo nel gioco, ciò include il numero totale di vittorie rispetto alle mani giocate, la somma delle vincite e delle perdite ottenute e capacità di intraprendere azioni ottimali in mani rischiose.  
    \item[Enviroment:] Il tavolo di Blackjack con:
    - mazzo di carte (stocastico)
    - dealer a policy fissa
    - mano del giocatore
    - regole del gioco.
    Le proprietà dell'ambiente sono le seguenti: -Parzialmente osservabile: l'agente è a conoscenza solo delle carte che ha attualmente in mano e le carte scoperte del dealer, non è a conoscenza dell'ordine delle carte rimanenti nel mazzo.
    Stocastico: l'ordinamento del mazzo è pseudocasuale e la distribuzione di esse è casuale.
    Sequenziale: ogni azione influenza l'esito finale.
    Statico: l'ambiente è invariato mentre un agente sta deliberando.
    Discreto: sebbene consideriamo più di un mazzo, il numero delle carte è pur sempre finito, di conseguenza abbiamo un numero finito sia di azioni che di stati.
    Singolo agente: il dealer opera seguendo una politica ben precisa, l'agente è l'unico "agente decisionale". 
    \item[Attuatori:] le azioni che può intraprendere l'agente sono quelle di Hit, Stand, Double e Split sopracitate.
    \item[Sensori:] - "Occhio virtuale" sul tavolo: percepisce le carte visibili nella propria mano e le carte scoperte del dealer.
    - Contatore di puntate:  rileva la propria puntata corrente e quella del dealer.
    - Segnalatore di possibilità di azione: informa l’agente se può eseguire azioni come come double e split.
    - Segnalatore di esito della mano: rileva i casi di vittoria, sconfitta e pareggio.
    Abbiamo identificato ulteriori sensori che potrebbero essere di aiuto ma in questa fase di sviluppo non siamo totalmente sicuri della loro utilità:
    - Contatore sul mazzo: determina il numero di carte rimaste nel mazzo.
\end{description}
\end{document}
